\documentclass{article}
\usepackage{libertine}
% \usepackage[libertine]{newtxmath}
\usepackage{tlatex}
\usepackage{lineno}
\pagenumbering{gobble}
\usepackage{color}
\definecolor{boxshade}{gray}{0.85}
\setboolean{shading}{true}

\begin{document}

\begin{linenumbers}
    \setcounter{linenumber}{20}

\begin{pcal}
l1: \* do a first collect:
        while (read # R) with (r \in R \ read) {
            collect[r] := registers[r];
            active := active \cup collect[r].active;
            read := read \cup {r} };
        read := {}; \* reset read
        snapshot := TRUE; \* true until we read a different value
l2: \* do a second collect:
        while (read # R) with (r \in R \ read) {
            if (registers[r] # collect[r])
                snapshot := FALSE;
            collect[r] := registers[r];
            active := active \cup collect[r].active;
            read := read \cup {r} };
        read := {}; \* reset read
l3: \* if more than K processes are active, output:
        if (Cardinality(active) > K)
            output := active;
l4: \* if the snapshot did not succeed, retry:
        if (\neg snapshot) goto l1;
l5: \* if the snapshot succeeded and the process's active set is everywhere, output:
        if (\A r \in R : collect[r].active = active)
            output := active; }}}
\end{pcal}

\end{linenumbers}

\end{document}


