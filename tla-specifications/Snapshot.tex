\documentclass{article}
\usepackage{libertine}
% \usepackage[libertine]{newtxmath}
\usepackage{tlatex}
\usepackage{lineno}
\pagenumbering{gobble}
\usepackage{color}
\definecolor{boxshade}{gray}{0.85}
\setboolean{shading}{true}

\begin{document}

\begin{linenumbers}

\begin{pcal}
--algorithm Snapshot {
    variables
        \* the shared, anonymous registers, each containing a view and a level:
        register = [r \in R |-> [view |-> {}, level |-> 0]];
    process (proc \in P)
        variables \* local variables:
            input \in V; \* the input of the process
            view = {input}; \* the set of all inputs ever read
            level = 0;
	    read = [r \in R |-> <<>>]; \* stuff read in the current scan loop
            written = {}; { \* registers written since this was last empty
	    output = <<>>;
write:  while (level < N-1) {
            with (r \in R \ written) { \* pick a register that we have not yet written to
                register[r] := [view |-> view, level |-> level];
                written := IF written \cup {r} = P THEN {} ELSE written \cup {r} };
	    read := [r \in R |-> <<>>]; \* reset stuff read
scan:       with (notRead = {r \in R : read[r] = <<>>})
	    while (notRead # {}) with (r \in notRead)
                read[r] := register[r];
            if (\A r \in R : read[r].view = view)
                level := Min({read[r].level : r \in R}) + 1
            else
                level := 0;
            view := view \cup (UNION {read[r].view : r \in R}) };
output: output := view } }
\end{pcal}

\end{linenumbers}

\end{document}

